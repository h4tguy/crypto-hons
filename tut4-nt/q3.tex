\begin{enumerate}

\item Note that 24 is a quadratic residue mod 55 if and only if 24 is a
quadratic residue both mod 11 and mod 5 \cite{slides_99}.

However, $24 \equiv 2 \textrm{ mod } 11$ and
$2^{(11-1)/2} \equiv 10 \textrm{ mod } 11$, so by Euler's criterion, 24 is not
a quadratic residue mod 55.

\item Note that $11 \equiv 3 \textrm{ mod } 4$, so the square roots of 3 are
given by \cite{slides_92}
\begin{eqnarray*}
&& \pm 3^{(11+1)/4} \textrm{ mod } 11 \\
&=& \pm 27 \textrm{ mod } 11 \\
&=& \pm 5 \textrm{ mod } 11 \\
\end{eqnarray*}

So the square roots of $3$ mod $11$ are the integers of the form $5 + 11k$ and
$-5 + 11k$, $\forall k \in \mathbb{Z}$.

\item Assuming that $14$ has square roots modulo $55$, and given that
$55 = 5 \cdot 11$ and $14$ is coprime to $55$, we know that $14$ must have $4$
square roots of the form
\begin{eqnarray*}
&& \pm b(11^{-1} \textrm{ mod } 5)11 \pm c(5^{-1} \textrm{ mod } 11)5
\textrm{ mod } 55 \\
&=& \pm b(1)11 \pm c(9)5 \textrm{ mod } 55 \\
&=& \pm 11b \pm 45c \textrm{ mod } 55
\end{eqnarray*}
where $b$ and $c$ are the square roots of $14$ modulo $5$ and $11$ respectively.
\cite{slides_100}

The square roots of $14$ modulo $5$ are congruent to the square roots of $4$
modulo $5$, which are $\pm 3 \textrm{ mod } 5$, by inspection.

The square roots of $14$ modulo $11$ are congruent to the square roots of $3$
modulo $11$, which were found to be $\pm 5 \textrm{ mod } 11$ in the previous
question.

So the square roots of $14 \textrm{ mod } 55$ are
\begin{eqnarray*}
&& \pm 11 \cdot 3 \pm 45 \cdot 5 \textrm{ mod } 55 \\
&=& \pm 33 \pm 225 \textrm{ mod } 55 \\
&\equiv& \pm 33 \pm 5 \textrm{ mod } 55 \\
\end{eqnarray*}

And $-33 = 22 \textrm{ mod } 55$, so the square roots are $38, 28, 27$ and
$17$.

\item Since $2$ is given as a generator of $\mathbb{Z}_{11}^{*}$, we know that
all of the even powers of $2$ are going to be the quadratic residues modulo
$11$, and all of the odd powers of $2$ are going to be the quadratic nonresidues
modulo $11$. It follows that $x$ is even if and only if $7$ is a quadratic
residue modulo $11$. Applying Euler's criterion, we find that $7^5 \equiv 10
\textrm{ mod } 11$, so $7$ is not a quadratic residue and therefore $x$ must be
odd.

\end{enumerate}
