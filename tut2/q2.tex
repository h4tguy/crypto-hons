\usepackage{amsfonts}

\begin{enumerate}

\item if $\underline{s}$ is a maximum length lfsr sequence of an lfsr of length
$n$, then it has period $2^n-1$ (as covered in lectures). this means that the
lfsr goes through $2^n-1$ distinct states before repeating a state.
specifically, these states are the members of $\mathbb{f}_2^n \setminus
(0,\dots,0)$. also, the first bits of these states are the bits of the sequence
in a single period.

to check whether $\underline{s}$ is balanced, we just need to check whether the
first bits of the members of $\mathbb{f}_2^n \setminus (0,\dots,0)$ (the states)
are balanced. clearly, there are $2^{n-1}$ states that start with 1 and
$2^{n-1}-1$ states that start with 0 (since including (0,\dots,0) as a state
implies that it is the only state). since the number of bits are odd and the
zero state is not included, we would expect a balanced sequence to have one more
1 than 0, which is what we have.

\end{enumerate}
