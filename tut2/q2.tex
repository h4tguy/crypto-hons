\begin{enumerate}
\item
\item
\item We will prove the result with linear algebra.

\paragraph{Lemma:} Let $k$ be an integer with $1 \le k < 2^L-1$ and let $T$
be a linear transformation $\mathbb{F}^L_2 \to \mathbb{F}^L_2$ which
generates an m-sequence $(s_i)_{i=1}^\infty$. Then $\text{ker}(T^k+\mathbb{I})=\{0\}$.

\paragraph{Proof of Lemma:} Define $V$ as $\text{im}(T^k+\mathbb{I})$ and let
$c_T$ be the characteristic (and, in this case, minimal) polynomial of $T$.
Suppose this were not the case, that
$\text{dim ker}(T^k+\mathbb{I})>0$. Then $\text{dim}\ V<L$ by Rank-Nullity Theorem.
But $V$ is $T$ invariant, since, for $v \in \mathbb{F}_2^L$,
$T(T^k+\mathbb{I})v=(T^k+\mathbb{I})Tv$,
so let $P$ be the characteristic polynomial of $T$ restricted to $V$.
If $v$ is a vector in $\mathbb{F}^L_2$, then note $P(T)(T^k+\mathbb{I})v=0$,
by Cayley-Hamilton theorem on $V$. So $P(T)(T^k+\mathbb{I})=0$, and hence
$c_T(x) | P(x)(x^k+1)$ ($c_T$ is the minimal polynomial of the sequence
since it is irreducible). But $c_T$ is irreducible since $T$ generates an
m-sequence, and $\deg P<L=\deg c_T$, so $c_T(x) | x^k+1$, which implies
$k\ge 2^L-1$ (\cite{slides}, Slide 37), a contradiction. This concludes
the proof of the lemma.

\paragraph{Main Proof:} The $2^L-1$ $L$-vectors starting at each position
in an m-sequence consist of every non-zero $L$-vector. Since 
$\text{ker}(T^k+\mathbb{I})=\{0\}$, these vectors are permuted in this map.
Specifically, there are $2^{L-1}-1$ vectors starting with $0$ and $2^{L-1}$ vectors
starting with $1$ in the image of this map. However, $s_x$ and $s_{x+k}$
disagree iff $s_{x+k}\oplus s_x=1$, or iff $T^k{\bf s_x}+{\bf s_x}$ begins with
$1$, where ${\bf s_x}$ is the $L$-vector starting with $s_x$ in the m-sequence.
However, the probability of $T^k{\bf s_x}+{\bf s_x}=(T^k+\mathbb{I}){\bf s_x}$
starting with $1$ is approximately $1/2$, since $T^k+\mathbb{I}$ maps about half the non-zero
$L$-vectors to vectors starting with $0$ and about half to vectors starting with
$1$. So the m-sequence
is approximately uncorrelated with itself shifted $k$ places whenever $k$ is not
a multiple of $2^L-1$.
\end{enumerate}
