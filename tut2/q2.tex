\begin{enumerate}

\item If $\underline{s}$ is a maximum length LFSR sequence of an LFSR of length
$L$, then it has period $2^L-1$ (as covered in lectures). This means that the
LFSR goes through $2^L-1$ distinct states before repeating a state.
Specifically, these states are the members of $\mathbb{F}_2^L \setminus
(0,\dots,0)$. Also, the first bits of these states are the bits of the sequence
in a single period.

To check whether $\underline{s}$ is balanced, we just need to check whether the
first bits of the members of $\mathbb{F}_2^L \setminus (0,\dots,0)$ (the states)
are balanced. Clearly, there are $2^{L-1}$ states that start with 1 and
$2^{L-1}-1$ states that start with 0 (since including $(0,\dots,0)$ as a state
implies that it is the only state). Thus, the number of 0s and 1s differ by one
and therefore the sequence is balanced.

\item In a maximum length LFSR, the state of the LFSR register runs through all
of the members of $\mathbb{F}_2^L \setminus (0,\dots,0)$ in a single period
(\cite{slides}, Slide 36). A run of length $i$ occurs at a time $t$ with state
$s_t$ if the preceding state $s_{t-1}$ (at time $t-1$) starts with a known
string of $i+2$ bits (eg. $i$ 1s between two 0s or $i$ 0s between two 1s). There
are $2^{L-i-2}$ such states that precede the start of a run of i 0s or 1s.
Accounting for both runs of 1s and of 0s, the total number of states that
precede the start of any run is $2 \times 2^{L-i-2} = 2^{L-i-1}$.

Any state that precedes the start of a run either starts with 01 or 10. This
shows that approximately half of the states precede the start of a run: the half
that begins with 01 or 10 rather than 11 or 00. If the total number of states is
$tot_{states} = 2^L-1$ (excluding (0,\dots,0) ), then the number of states that
precede the start of a run is $tot_{runs} = 2^{L-1}$ - half of the bit strings
of length L.

This shows that the number of runs of length $k$ for $1 \le k \le L-1$ is

\[2^{L-1-k} = 2^{-k} \times 2^{L-1}\]

which means that $2^{-k}$ of all runs are of length k.

This breaks down when you consider runs of length L. However, there can only be
one run of length L: the run of L 1s. The run of L 0s is the 0 state, and is
never reached in a maximum-length LFSR whose initial state is non-zero, as only
non-zero states are found in such an LFSR (\cite{slides}, Slide 37).

\end{enumerate}
