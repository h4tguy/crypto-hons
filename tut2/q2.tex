\begin{enumerate}

\item If $\underline{s}$ is a maximum length LFSR sequence of an LFSR of length
$L$, then it has period $2^L-1$ (as covered in lectures). This means that the
LFSR goes through $2^L-1$ distinct states before repeating a state.
Specifically, these states are the members of $\mathbb{F}_2^L \setminus
{(0,\dots,0)}$. Also, the first bits of these states are the bits of the
sequence in a single period.

To check whether $\underline{s}$ is balanced, we just need to check whether the
first bits of the members of $\mathbb{F}_2^L \setminus {(0,\dots,0)}$ (the
states) are balanced. Clearly, there are $2^{L-1}$ states that start with 1 and
$2^{L-1}-1$ states that start with 0 (since including $(0,\dots,0)$ as a state
implies that it is the only state). Thus, the number of 0s and 1s differ by one
and therefore the sequence is balanced: given a large $L$, a randomly chosen bit
will be 0 or 1 each with probability approximately $1/2$.

\item In a maximum length LFSR, the state of the LFSR register runs through all
of the members of $\mathbb{F}_2^L \setminus {(0,\dots,0)}$ in a single period
(\cite{slides}, Slide 36). A run of length $i$ occurs at a time $t$ with state
$s_t$ if the preceding state $s_{t-1}$ (at time $t-1$) starts with a known
substring of $i+2$ bits (eg. (0, 1, \dots, 1, 0) - 1 repeated $i$ times - or (1,
0, \dots, 0, 1) (0 repeated $i$ times)). There are $2^{L-i-2}$ such states that
precede the start of a run of $i$ 0s or 1s. Accounting for both runs of ones and
of zeros, the total number of states that precede the start of any run is $2
\times 2^{L-i-2} = 2^{L-i-1}$.

Any state that precedes the start of a run either starts with 01 or 10. This
shows that approximately half of the states precede the start of a run: the half
that begins with 01 or 10 rather than 11 or 00. If the total number of states is
$tot_{states} = 2^L-1$ (excluding (0,\dots,0) ), then the number of states that
precede the start of a run is $tot_{runs} = 2^{L-1}$ - half of the bit strings
of length L.

This shows that the number of runs of length $k$ for $1 \le k \le L-1$ is

\[2^{L-1-k} = 2^{-k} \times 2^{L-1}\]

which means that $2^{-k}$ of all runs are of length k.

This breaks down when you consider runs of length L. However, there can only be
one run of length L: the run of L 1s. The run of L 0s is the 0 state, and is
never reached in a maximum-length LFSR whose initial state is non-zero, as only
non-zero states are found in such an LFSR (\cite{slides}, Slide 37).

\item We will prove the result with linear algebra.

\paragraph{Lemma:} Let $k$ be an integer with $1 \le k < 2^L-1$ and let $T$
be a linear transformation $\mathbb{F}^L_2 \to \mathbb{F}^L_2$ which
generates an m-sequence $(s_i)_{i=1}^\infty$. Then $\text{ker}(T^k+\mathbb{I})=\{0\}$.

\paragraph{Proof of Lemma:} Define $V$ as $\text{im}(T^k+\mathbb{I})$ and let
$c_T$ be the characteristic (and, in this case, minimal) polynomial of $T$.
Suppose this were not the case, that
$\text{dim ker}(T^k+\mathbb{I})>0$. Then $\text{dim}\ V<L$ by Rank-Nullity Theorem.
But $V$ is $T$ invariant, since, for $v \in \mathbb{F}_2^L$,
$T(T^k+\mathbb{I})v=(T^k+\mathbb{I})Tv$,
so let $P$ be the characteristic polynomial of $T$ restricted to $V$.
If $v$ is a vector in $\mathbb{F}^L_2$, then note $P(T)(T^k+\mathbb{I})v=0$,
by Cayley-Hamilton theorem on $V$. So $P(T)(T^k+\mathbb{I})=0$, and hence
$c_T(x) | P(x)(x^k+1)$ ($c_T$ is the minimal polynomial of the sequence
since it is irreducible). But $c_T$ is irreducible since $T$ generates an
m-sequence, and $\deg P<L=\deg c_T$, so $c_T(x) | x^k+1$, which implies
$k\ge 2^L-1$ (\cite{slides}, Slide 37), a contradiction. This concludes
the proof of the lemma.

\paragraph{Main Proof:} The $2^L-1$ $L$-vectors starting at each position
in an m-sequence consist of every non-zero $L$-vector. Since 
$\text{ker}(T^k+\mathbb{I})=\{0\}$, these vectors are permuted in this map.
Specifically, there are $2^{L-1}-1$ vectors starting with $0$ and $2^{L-1}$ vectors
starting with $1$ in the image of this map. However, $s_x$ and $s_{x+k}$
disagree iff $s_{x+k}\oplus s_x=1$, or iff $T^k{\bf s_x}+{\bf s_x}$ begins with
$1$, where ${\bf s_x}$ is the $L$-vector starting with $s_x$ in the m-sequence.
However, the probability of $T^k{\bf s_x}+{\bf s_x}=(T^k+\mathbb{I}){\bf s_x}$
starting with $1$ is approximately $1/2$, since $T^k+\mathbb{I}$ maps about half the non-zero
$L$-vectors to vectors starting with $0$ and about half to vectors starting with
$1$. So the m-sequence
is approximately uncorrelated with itself shifted $k$ places whenever $k$ is not
a multiple of $2^L-1$.

\end{enumerate}
