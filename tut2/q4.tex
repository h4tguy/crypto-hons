\begin{enumerate}

\item A LFSR of length $L > 1$ with irreducible characteristic polynomial $f(X)$,
has $2^{L}-1$ unique non-zero states. If the LFSR has initial state of zero it produces a
cycle of period 1, which divides $2^{L}-1$. For each non-zero initial state the LFSR produces a sequence with period of at most $2^{L}-1$. However we know that each state has exactly one directed edge coming into it so 
each state can only appear in one sequence. Thus if the period of a sequence produced
by any non-zero initial state is less than $2^{L}-1$, the LFSR state diagram 
consists of $n$ disjoint cycles, where $n \geq 2$. We know that these cycles 
must all be of the same period, $p$. Since the number of possible states is fixed 
and each state must appear in exactly one cycle, we know that $n \times p = 2^{L}-1$.
Thus the period of the LFSR for every non-zero initial state divides $2^{L}-1$.

\item Let $p$ be an irredicuible polynomial of degree $5$ and let $s$ be a non-trivial
sequence generated by $p$. Then the period of $s$ divides $2^5-1=31$, so is $1$ or $31$.
But $s$ is non-trivial, so the period is $31$. This is the order of its minimal polynomial
(\cite{slides}, Theorem 2), which must be $p$ since $p$ is an irreducible characteristic
polynomial. So the order of $p$ is $2^5-1$, and $p$ is primitive.

\end{enumerate}
