\begin{enumerate}

\item If a LFSR has an irreducible characteristic polynomial $f(X)$, of length $L > 1$, then the minimal polynomial of the LFSR must either be $f(X)$ or $1$ (\cite{slides}, Theorem 1). If the minimal polynomial of the LFSR is $1$ then it has a period of 1, which divides $2^{L}-1$. If the minimal polynomial is $f(X)$ then the period of the LFSR is equal to the order of the minimal polynomial (\cite{slides}, Theorem 2). Furthermore we know that $f(x)$ divides $X^{P} - 1$ (\cite{slides}, Theorem 1), which implies that the order of $f(x)$ divides the order of $X^{P} - 1$. We know that the order of $X^{P} - 1$ is $2^{L} - 1$ and therefore the period of $f(X)$ divides $2^{L} - 1$.

\item Let $p$ be an irredicuible polynomial of degree $5$ and let $s$ be a non-trivial
sequence generated by $p$. Then the period of $s$ divides $2^5-1=31$, so is $1$ or $31$.
But $s$ is non-trivial, so the period is $31$. This is the order of its minimal polynomial
(\cite{slides}, Theorem 2), which must be $p$ since $p$ is an irreducible characteristic
polynomial. So the order of $p$ is $2^5-1$, and $p$ is primitive.

\end{enumerate}
