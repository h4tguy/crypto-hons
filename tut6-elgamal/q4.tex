First in order to solve the DLP for:
\begin{eqnarray}
    49^{x} \equiv 144 \mod 151
\end{eqnarray}

We need to find the prime factors of the order of $49$ in the group 
$\mathbb{Z}^{*}_{151}$. They are $N = 75 = 3 \cdot 5^{2}$.

Raising both sides of equation (1) to $5^{2}$, will give us a DLP in a subgroup of order 3, which is prime.
\begin{eqnarray*}
    (49^{5^{2}})^{x} \equiv 144^{5^{2}} \mod 151 \\
    32^{x} \equiv 118 \mod 151
\end{eqnarray*}

Since the order of 32 is 3, this DLP can be trivially solved using brute force yielding:
\begin{eqnarray}
    x \equiv 2 \mod 3
\end{eqnarray}


Raising both sides of equation (1) to $3$, will give us a DLP in a subgroup of order 25, which is not prime.
\begin{eqnarray*}
    (49^{3})^{x} \equiv 144^{3} \mod 151 \\
    20^{x} \equiv 110 \mod 151
\end{eqnarray*}
In order to work in only subgroups of prime order we now have to write the unknown $x$ to the base 5:
\begin{eqnarray*}
    x = x_{0} + 5 \cdot x_{1}
\end{eqnarray*}
The DLP can now be written as:
\begin{eqnarray}
    110 \equiv 20^{x_0 + 5 \cdot x_{1}} \mod 151 \\
    20^{x_0} \cdot 20^{5 \cdot x_{1}} \mod 151
\end{eqnarray}

Raising equation (4) to 5, yields a DLP for $x_{0}$ in a group of order 5:
\begin{eqnarray*}
    110^{5} \equiv (20^{5})^{x_0} \mod 151 \\ 
    59 \equiv 8^{x_{0}} \mod 151
\end{eqnarray*}
Since the order of 8 is 5 it can be easily solved using brute force:
\begin{eqnarray}
    x_{0} \equiv 3 \mod 5
\end{eqnarray}

Multiplying (4) through by $20^{-x_{0}} \equiv 20^{-3} \equiv 20^{22} \equiv 50 \mod 151$ yields a DLP for $x_{1}$ in a group of order 5:
\begin{eqnarray*}
    110 \cdot 50 \equiv 20^{5 \cdot x_{1}} \mod 151 \\
    64 \equiv 8^{x_{1}} \mod 151
\end{eqnarray*}
Since the group order is 5 this DLP can also be solved trivially via brute force:
\begin{eqnarray}
    x_{1} \equiv 2 \mod 5
\end{eqnarray}

Using equations (5) and (6) we can now calculate the DLP for x in the subgroup of order 25:
\begin{eqnarray*}
    x &\equiv& x_{0} + 5 \cdot x_{1} \mod 25\\
    &\equiv& 3 + 5 \cdot 2 \mod 25 \\
    &\equiv& 13 \mod 25
\end{eqnarray*}

We can now use the Chinese Remainder Theorem to solve for x:
\begin{eqnarray*}
    x &=& 2 \cdot 25 \cdot (25^{-1} \mod 3) + 13 \cdot 3 \cdot (3^{-1} \mod 25) \mod 75 \\
    &=& 50 + 663 \mod 75 \\
    &=& 38
\end{eqnarray*}