Let's execute this method for $m = \sqrt{16} = 4$. We can find that $16^{-1} \equiv 19 \mod 101$ using the fact that $16$ has order $25 \mod 101$. We then undertake the Baby-step Giant-step method, as detailed in Table \ref{tab:baby} and Table \ref{tab:giant}. 

From our calculations, we can see that $52 \times 16^{-1} \equiv 16^{16} \mod 101$. This means that

\[16^{16+1} \equiv 16^{17} \equiv 52 \mod 101.\] Therefore, $x=17$.

\begin{table}
\centering
\begin{tabular}{l l}
r & $52(16^{-1})^{r} \mod 101$ \\\hline
0 & 52 \\\hline
1 & 79 \\\hline
2 & 87 \\\hline
3 & 37 \\\hline
\end{tabular}
\caption{Baby steps for the Baby-step Giant-step method in Question 2}
\label{tab:baby}
\end{table}

\begin{table}
\centering
\begin{tabular}{l l}
i & $16^{ 4 i} \mod 101$ \\\hline
0 & 1 \\\hline
1 & 88 \\\hline
2 & 68 \\\hline
3 & 25 \\\hline
4 & 79 \\\hline
\end{tabular}
\caption{Giant steps for the Baby-step Giant-step method in Question 2}
\label{tab:giant}
\end{table}

\begin{figure}[P]
\caption{Baby-step Giant-step method Code}
\lstinputlisting[language=Python]{q2_helper.py}
\end{figure}
