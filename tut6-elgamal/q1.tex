\begin{enumerate}
\item Alice's signature consists of the numbers $r \equiv g^k\mod p$ and
  $s \equiv k^{-1}(H(M) - ar)\mod (p-1)$
  \begin{align*}
    r &\equiv 2^{11} \mod 101 \\
      &\equiv 28 \mod 101 \\
    s &\equiv 11^{-1}(6 - 42 \cdot 28)\mod 100 \\
      &\equiv 91 \cdot 30\mod 100 \\
      &\equiv 30 \mod 100\\
  \end{align*}
  As $11^{-1} \equiv 91\mod 100$, by the extended Euclidean algorithm. Therefore,
  Alice's signature is $(28, 30)$

\item Alice's public key consists of $p = 101$, $g = 2$ and $g^a \mod p = 2^{42}
\mod 101 \equiv 43 \mod 101$, so her public key is $(101, 2, 43)$.

  To verify Alice's signature, Bob computes $g^{H(M)} \mod p \equiv 2^6 \mod 101
  \equiv 64 \mod 101$ and compares it with
  \begin{align*}
    (g^a)^r \cdot r^s \mod p &\equiv 43^{28} \cdot 28^{30} \mod 101 \\
                             &\equiv 81 \cdot 17 \mod 101 \\
                             &\equiv 64 \mod 101\\
  \end{align*}
  These numbers are equivalent modulo $101$, so Bob has verified Alice's signature
  .

\item Given that $s \equiv k^{-1}(H(M) - ar)\mod (p-1)$, we know that
  \begin{align*}
    ar &\equiv (H(M) - sk) \mod (p-1) \\
    28a &\equiv 6 - 30 \cdot 11 \mod 100 \\
    28a &\equiv 76 \mod 100 \\
    7a &\equiv 19 \mod 25 \\
  \end{align*}

  But we can use the extended Euclidean algorithm to find that $7^{-1} \mod 25
  \equiv 18 \mod 25$, which gives:
  \begin{align*}
    a &\equiv 18 \cdot 19 \mod 25 \\
      &\equiv 17 \mod 25 \\
  \end{align*}
  Since $a \equiv 17 \mod 25$, it must be equivalent to either $17$, $42$, $67$ or
  $92$ modulo $100$. Alice's public key tells us that $g \equiv 2 \mod 101$ and
  that $g^a \equiv 43 \mod 101$, so we can try each of our $4$ values for $a$.
  Doing this shows that $a \equiv 43 \mod 101$ satisfies $2^a \equiv 43 \mod 101$.

\end{enumerate}
