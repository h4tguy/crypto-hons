\begin{enumerate}

\item
\begin{tabular}{l l l}
\hline$j$ & $3^{j!} \text{ mod } 323$ & $\text{gcd}(3^{j!} - 1  \text{ mod } 323, 323)$ \\ \hline
2 & 9 & 1 \\ \hline
3 & 83 & 1 \\ \hline
4 & 254 & 1 \\ \hline
5 & 220 & 1 \\ \hline
6 & 1 & 323 \\ \hline
\end{tabular}

Therefore the Pollard $p-1$ method fails to factorise 323 with a base of 3.

\item
\begin{tabular}{l l l}
\hline$j$ & $2^{j!} \text{ mod } 323$ & $\text{gcd}(2^{j!} - 1  \text{ mod } 323, 323)$ \\ \hline
2 & 4 & 1 \\ \hline
3 & 64 & 1 \\ \hline
4 & 273 & 17 \\ \hline
\end{tabular}

Therefore 17 is a factor of 323. This means $323 = 17 \times 19$.

\item We failed to factorise 323 with a base of 3 as $3^{6!} \equiv 1 \text{ mod } 323 $. As shown in lectures, this occurs when the smallest $k$ st. order of 3 in $\mathbb{Z}_p^*$ divides $k!$ is the same for every prime factor of $323$. Since we now know the prime factors, we can verify this. The order of $3$ in $\mathbb{Z}_{17}^*$ and $\mathbb{Z}_{19}^*$ is 16 and 18 respectively. The smallest $k$ such that $16 | k!$ and $18 | k!$ is 6 as shown below.

\begin{tabular}{l l l}
\hline k & $k! \text{ mod } 16$ & $k! \text{ mod } 18$ \\ \hline
2 & 2 & 2 \\ \hline
3 & 6 & 6 \\ \hline
4 & 8 & 6 \\ \hline
5 & 8 & 12 \\ \hline
6 & 0 & 0 \\ \hline
\end{tabular}

As they share the same smallest $k$, this method fails. However, this method works for a base of $2$. To see why, we repeat the same process as above. The order of $2$ in $\mathbb{Z}_{17}^*$ and $\mathbb{Z}_{19}^*$ is 8 and 18 respectively. As shown in the table below, the smallest $k$ such that $8 | k!$ is 4 while the smallest $k$ such that $18 | k!$ is 6. 

\begin{tabular}{l l l}
\hline k & $k! \text{ mod } 8$ & $k! \text{ mod } 18$ \\ \hline
2 & 2 & 2 \\ \hline
3 & 6 & 6 \\ \hline
4 & 0 & 6 \\ \hline
5 & 0 & 12 \\ \hline
6 & 0 & 0 \\ \hline
\end{tabular}

as $4 \neq 6$, we are able to factorise $323$ with a base of $2$.
\end{enumerate}
