\begin{enumerate}
\item Alice's public key is $(n,e)=(pq,e)=(437,13)$. Her private key is
$d$, the inverse of $e$ modulo $\phi(n)=(p-1)(q-1)=396$. But
\begin{align*}
	396 &= 1 \cdot 396 + 0 \cdot 13 \\
	13 &= 0 \cdot 396 +1 \cdot 13 \\
	6 &=  1 \cdot 396 - 30 \cdot 13 \\
	1 &= -2 \cdot 396 + 61 \cdot 13
\end{align*}
(where $396=30\cdot 13 + 6$ and $13=2 \cdot 6 + 1$). It hence
follows that $61 \cdot 13 \equiv_{396} 1$, so Alice's private key is $d=61$.

\item Alice must compute $H(M)^d$ mod  $n$, which is $6^{61}$ mod $437$.
Now $61=2^5+2^4+2^3+2^2+2^0$. Also, since $6^{2^k} \equiv_{437} \left(6^{2^{k-1}}\right)^2$,
\begin{align*}
	6^{2^0} &\equiv_{437} 6 \\
	6^{2^1} &\equiv_{437} 36 \\
	6^{2^2} &\equiv_{437} 422 \\
	6^{2^3} &\equiv_{437} 225 \\
	6^{2^4} &\equiv_{437} 370 \\
	6^{2^5} &\equiv_{437} 119
\end{align*}
And so Alice's signature is 
\begin{align*}
S=6^{61}&\equiv_{437}6^{32} \cdot 6^{16} \cdot 6^8 \cdot 6^4 \cdot 6^1 \\
&\equiv_{437} 119 \cdot 370 \cdot 225 \cdot 225 \cdot 422 \cdot 6 \\
&\equiv_{437} 104
\end{align*}
(which she publishes).

\begin{figure}
\caption{Square and multiply code}
\lstinputlisting[language=Python]{square_and_multiply.py}
\end{figure}

\item Bob may not know $d$, but he knows that $H(M)^{d'e} \equiv_{n} H(M)$
iff $\text{ord}_{n} H(M)|d'e-1$. Since $\phi(n)$ is typically large,
an attacker cannot efficiently find a $d'$ for which this is the case.
Since Alice's $d$ satisfies this property (since $\text{ord}_{n} H(M) |\phi(n)$),
Bob can compute $S^e \equiv_n H(M)^{de}$, which will be $H(M)$ mod $n$ iff
the signature is valid.

\item We need to compute $6^{61}$ mod $437$. We start by computing mod $19$.
$61 \equiv_{18} 7$ (where $18=\phi(19)=19-1$ since $19$ is prime). So we only
need to compute 
\begin{align*}
6^7 &\equiv_{19} 6^1 \cdot 6^2 \cdot 6^4 \\
	&\equiv_{19} 6 \cdot 17 \cdot 4 \\
	&\equiv_{19} 9.
\end{align*}
We do the same modulo $23$, noting $61 \equiv_{22} 17$.
\begin{align*}
6^{15} &\equiv_{23} 6^1 \cdot 6^{16} \\
 &\equiv_{23} 6 \cdot 2 \\
 &\equiv_{23} 12
\end{align*}
So, using Chinese Remainder Theorem, $6^{61}$ is the unique residue modulo $437$
which is congruent to $9$ mod $19$ and $12$ mod $23$. But this is just
$12 \cdot 19 \cdot a + 9 \cdot 23 \cdot b$ where $a$ is the inverse of $19$
mod $23$ and $b$ is the inverse of $23$ mod $19$. We use the Extended Euclidean
Algorithm to compute these
\begin{align*}
23 &= 1 \cdot 23 + 0 \cdot 19 \\
19 &= 0 \cdot 23 + 1 \cdot 19 \\
4 &= 1 \cdot 23 - 1 \cdot 19 \\
3 &= -4 \cdot 23 + 5 \cdot 19 \\
1 &= 5 \cdot 23 - 6 \cdot 19
\end{align*}
and so $a=-6$, $b=5$.
So then $6^{61} \equiv_{437} 12 \cdot 19 \cdot (-6) + 9 \cdot 23 \cdot 5 \equiv_{437} 104$
as expected.

\item In this case, the Chinese Remainder Theorem yields the bad signature
$S'\equiv_{437} 12 \cdot 19 \cdot (-6) + 73 \cdot 23 \cdot 5 \equiv_{437} 35$.

Bob will realise that $$(S')^e \equiv_{437} 35^{13} \equiv_{437} 328 \not\equiv_{437} 6 \equiv_{437} H(M)$$
If he guesses that one of the exponentiations in $\mathbb{Z}_p$ or $\mathbb{Z}_q$ was incorrect, but
the other was correct, wlog $(S')^e \equiv_q H(M)$, so $q=\text{gcd}(n,H(M)-(S')^e)$
(since $n$ does not divide $H(M)-(S')^e$. Indeed, in this case $H(M)-(S')^e \equiv_{437} 115$
and $\text{gcd}(115,437)=23$ yields a prime factor of $n$.

\end{enumerate}
