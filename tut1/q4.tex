%Collision resistance in a hash function means that finding two messages $m_1$
%and $m_2$ with the same hash $h = f(m_1) = f(m_2)$ requires, on average, as
%much computation as finding such a collision with a brute force attack (hashing
%messages at random until a collision is found). For a binary string of length
%$N$, there are $2^N$ possible strings that such a brute force attack would
%ever have to consider. However, due to the birthday paradox, on average an
%attacker will only have to find the hashes of $2^(N/2)$ messages before
%finding a collision. This means that a collision resistant function requires
%an attacker to hash an average of $2^(N/2)$ messages before they find a
%collision.

%4.a - contrapositive
\begin{enumerate}
\item Consider the contrapositive. If a hash function is not second-preimage
  resistant, then, when given an arbitrary hash value and one of its preimages,
  we can find a second preimage faster than a brute force
  search.

If this hash function is not collision resistant, then we can find two
messages with the same hash value faster than a brute force search (hashing
approximately $2^{N/2}$ messages for $n$-bit messages).

Given a hash function $h$ that isn't second preimage resistant, we can take
an $n$-bit message $m_1$ and its hash value $h(m_1)$ and find a second
message $m_2$ such that $h(m_1) = h(m_2)$. We now have two images that have
the same hash value (which gives us a collision), showing that $h$ is not
collision resistant.

%4.b - http://www.cs.usfca.edu/~ejung/courses/686/lectures/05hash.pdf
%citation for construction of the hash function g:
%https://drive.google.com/open?id=1-NRjCCJVj7NJnGK1Qy80pXTqcKsOBtrTsh6lp6af5r8&
%authuser=0
\item Given an $n-1$-bit collision resistant hash function $h$, construct
	another hash function \cite{jung} \\
  $g(x) = \left\{
    \begin{array}{lr}
      0|x & : \text{if x is length }n-1\\
      1|h(x) & : \text{otherwise}
    \end{array}
    \right\}$ \\
    %if this is too ugly, I could try cases from amsmath:
    %http://tex.stackexchange.com/questions/74129/piecewise-function

Now, any value produced by this function that starts with 0 is easily
invertible - its preimage is the last $n-1$ bits of the hash value. A hash
value whose first bit is 1 might not be easily invertible, but at least half
of the possible hash values are easily invertible (specificaly the half whose
first bit is 0). Despite the ease with which these preimages can be found,
collisions are still difficult to find. The hash values that start with 0 have
no collisions (by definition, as each corresponds to a unique $n-1$ bit string
), while those that start with 1 have as many collisions as $g(x)$. Since $g$ is
collision resistant, the hash values that $h$ generates either have no
collisions or are difficult to find, making $h$ collision resistant.

%4.c - not one-way => can generate preimage set given hash => can generate
%all other preimages given hash and preimage
\item Given a hash function $f$ that is not one-way, we can easily find a member
$m$ of the preimage of a given hash value $f(m)$. If we were given a message
$m_1$ and its hash value $f(m_1)$, we would expect to be able to find a second
preimage $m_2$, as we can easily find preimages. This doesn't apply to the
function in (b), because that function is collision resistant and therefore
second-preimage resistant (by (a)), but isn't one-way. Also, the part of the
function that is easily invertible only gives singleton preimages, so there is
no second member of the preimage to be found.

\end{enumerate}
