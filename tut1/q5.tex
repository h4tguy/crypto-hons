\begin{enumerate}
\item All permutations $f$ are invertible. In particular,  they are injective, so
$f(x)=f(y) \Rightarrow x=y$ and hence there are no collisions. They still have
cycles though. In fact, every value is in a cycle, so Floyd's cycle finding algorithm
will find exactly the length of the cycle a particular value is on, instead of
just a multiple of it. The second part of the attack, that designed to use the
cycle to find collisions, will fail since there are no collisions to find.

\item We assume that a meaningful message followed by some random data is
still meaningful. This is reasonable because adding randomness to messages to
increase their entropy is considered good practice. Let $H$ be the hash function
one is attempting to find a collision for, and let $m_0$ and $m_1$ be two
meaningful messages that one wishes to find a collision for. Let $\|$ be the
concatenation operator and let $a\%b$ be the least postive member of the residue
class of $a$ modulo $b$. Let $g_0=m_0\|r$, where $r$ is a random bit string.
Let $g_{i+1}=G(g_i)=H(m_{g_i\%2}\|g_i)$. Then Floyd's cycle finding attack will
find $i,j$ with $G(g_i)=G(g_j)$ in $\mathcal{O}(2^{n/2})$ time. With probability
$50\%$ (since $H$ is equidistributed over even and odd numbers), $g_i\not\equiv_2 g_j$
and hence $H(m_0\|g_{i'})=G(g_{i'})=G(g_{j'})=H(m_1\|g_{j'})$ where $i',j'$ are
a permutation of $i,j$. In this case we have found a collision between messages
with prefixes of our choice and we are done. Otherwise we repeat the algorithm
with a different value of $r$ until we succeed. Since the probability of success
on each iteration is $1/2$, the expected number of iterations is
\begin{eqnarray*}
	\frac{1}{2}+\frac{2}{4}+\frac{3}{8}+\dots = 2
\end{eqnarray*}
Hence the expected number of iterations is $\mathcal{O}(1)$ and the algorithm
takes $\mathcal{O}(2^{n/2})$ time (and $\mathcal{O}(n)$ space, like the original
cycle finding algorithm).

\end{enumerate}
