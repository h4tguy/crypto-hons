\begin{enumerate}
\item Let $H_n$ be the space of $n$-bit words and let $M$ be a space of message
blocks (typically $k$-bit words for $k>n$). Let $f: H_n \times M \rightarrow H_n$
be a function with an efficient inverse (given $m \in M$ and
$h \in H_n$, there is an efficient way of computing an $h'$ such that $f(h',m)=h$).
Also suppose that the values of this inverse function, as a function of $m$,
are randomly distributed in $H_n$.

We demonstrate an $\mathcal{O}(2^{n/2})$ -time and -space attack on a
Merkle-Damgard hash function using $f$ as a compression function. Let $M^*$ be
the set of arbitrary-length tuples of members of $M$. The hash function can
then be interpreted as a function $F: H_n \times M^* \rightarrow H_n$, where
\begin{eqnarray*}
F(h_0,(m_0,m_1,\dots,m_l))=f(f(\dots f(f(h_0,m_0),m_1)\dots,m_{l-1}),m_l).
\end{eqnarray*}
For a final hash value $h_{target}$ and an intialization vector $h_0$, our attack
finds a message $\nu$ with $F(h_0,\nu)=h_{target}$.

Our attack consists of two stages. In the first we determine a message
$\mu = (m_0,m_1,\dots,m_{2^{n/2}})$ and an initialisation vector (of our choice)
$\eta$, such that $f(\eta,\mu)=h_{target}$. The second stage finds a message $\nu$
such that $F(h_0,\nu)=F(\eta,\mu)$.

For the first stage, let $\mu_0$ be the empty message. Also set $g_0=h_{target}$. Then clearly 
$F(h_{target},\mu_0)=h_{target}$. We now generate sequences $(\mu_i),(g_i)$ of length $2^{n/2}$ 
such that at any stage $F(g_i,\mu_i)=h_{target}$, and the $g_i$ are distinct.
Given $\mu_i$ and $g_i$, in order to determine the next elements in the sequences,
choose a random message block $m'$. We can then efficiently find an $n$-bit word $g'$ with
$f(g',m')=g_i$. If $g'$ is distinct from each $g_j$ with $j\le i$, then set $g_{i+1}=g'$
and let $\mu_{i+1}$ be $\mu_i$ with $m'$ prepended. Then clearly 
\begin{eqnarray*}
	F(g_{i+1},\mu_{i+1})&=&F(f(g_{i+1},m'),\mu_i)\\
																		 &=&F(g_i,\mu_i)\\
																		 &=&h_{target}.
\end{eqnarray*}
Otherwise, choose a new $m'$
and repeat until $g'$ is distinct from the $g_j$ with $j\le i$. Since there are $i+1$ values
of $g_j$ and $2^n$ possible values of $g'$,  and the values of
the inverse function are randomly distributed, the probability of a collision is $\frac{i+1}{2^n}$.
But $i+1\lesssim 2^{n/2}$, so the probability of a collision at any stage
of the algorithm is small ($\mathcal{O}(2^{-n/2})$) and the expected number of collisions
when generating $2^{n/2}$ members of the sequence is $\mathcal{O}(1)$. Finally, set $\mu = \mu_{2^{n/2}}$
and $\eta=g_{2^{n/2}}$. Then $F(\eta,\mu)=h_{target}$, and the length of $\mu$ is $2^{n/2}$ as required.

For the second stage we simply use the long message attack. The message $\mu$
we have calculated is of length $2^{n/2}$ so the long message attack will take
$\mathcal{O}(2^{n-n/2})=\mathcal{O}(2^{n/2})$ time and space. This attack will
calculate a message $\nu=\nu'+\mu_i$, for some $i$, such that $F(h_0,\nu)=h_{target}$
as required. In particular, the message found at this stage will have the required
initialisation vector, as opposed to one of our choosing.

Finally, block ciphers have an efficient inverse if the key (in this case a block
of the message to be hashed) is known. So the attack above can be applied to block ciphers
if they are used as a compression function in a Merkle-Damgard hash function. However, if the encryption
takes the exclusive or of the result of the block cipher and the previous chaining
value, the inverse function can no longer be efficiently computed, and the attack fails.

\item $(h_i,m_{i+1})$ is a fixed point iff 
	\begin{eqnarray*}
		&&f(h_i,m_{i+1})=h_i \\
		&\Leftrightarrow& E_{m_{i+1}}(h_i)\oplus h_i = h_i \\
		&\Leftrightarrow& E_{m_{i+1}}(h_i)=0.
	\end{eqnarray*}
	So, for any value of $m_{i+1}$, then if $h_i=E^{-1}_{m_{i+1}}(0)$, then $(h_i,m_{i+1})$ is a fixed point.
\end{enumerate}
