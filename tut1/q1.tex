\begin{enumerate}
\item 
As we learnt in lectures\cite{Slides_43}, the probability of drawing an
individual ball at least twice twice after randomly drawing $k$ balls out of an
urn containing $m$ balls is \[1 - e^{-\frac{k^2}{2m}}\] for $m,k >> 0$
and $k << m$. 

If we assume that a day has 365 days and that birthdays are uniformly
distributed across the days of the year, then the probability that
2 people in a room of 30 share a birthday is approximately 
\[1 - e^{-\frac{30^2}{2 \times 365}} \approx 0.71 \]

We know that the odds of of drawing a particular ball out of an urn of $m$ balls
after $k$ tries 
 is \[1-e^{-\frac{k}{m}}\]\cite{Slides_36}. The chance that someone has a
 particular birthday (i.e. mine) in a room of 30 people is approximately \[1 -
 e^{-\frac{30}{365}} \approx 0.08\].

\item

If we number all 300 people from $p_1$ to $p_{300}$, then the probability that
$p_1$'s birthday is distinct from $p_2$ is $1-\frac{1}{365}$. Similarly, the
probability that $p_i$'s birthday is different from $p_j$'s for $j < i$ is $1 -
\frac{i}{365}$. This means that the probability that at least one person shares
a birthday with someone else in that room is 
\[1 - \prod^{299}_{i=0} \left(1 - \frac{i}{365} \right) \approx
6.24 \times 10^{-82}\]

The probability that any individual has my birthday is $\frac{1}{365}$, so the
probability that in a room of 300 that no-one has my birthday is \[\left(1 -
\frac{1}{365} \right)^{300}\]. Thus the probability that someone in a room of 300
shares my birthday is is \[1- \left(1 - \frac{1}{365} \right)^{300} \approx
0.56\].

\item
The chance that in a room with $k$ people, $2$ or more people share a birthday
in a calendar with $m$ days is $p$ given by, as per (a),
\begin{align*}
p &=1 - e^{-\frac{k^2}{2m}} \\
\Rightarrow \ln(1-p) &= - \frac{k^2}{2m} \\
\Rightarrow \sqrt{-2m\ln(1-p)} &= k
\end{align*}

For 2 people sharing a birthday with 75\% probability, we need
\[\sqrt{-(2 \times 365)\ln(1-0.75)} = \sqrt{730\ln(0.25)} \approx 31.81 \text{
People}\] 

\item

366 by the Pidgeon Hole Principle. 

Intuitively, if there are 365 or fewer people in a room, then every person could
have their own distinct birthday. Since this is possible with non-zero
probability, it isn't possible to say with 100\% certainty that two people will
share a birthday. With 366 people, two people will have to share a birthday as
there are only 365 birthdays to go round as per the Pidgeon Hole Principle.
\item 

As derived in (b), the probability that out of a random group of 356 people that one
will share my birthday is 
\[1 - \left(1 - \frac{1}{365}\right)^365 \approx 63\%\]

As per the notes\cite{Slides_46}, we have that it takes on average $m$ tries to
pick a particular ball randomly form an urn with $m$ balls. This means that I
should expect to ask 365 people on average before getting any particular fixed
birthday.
\end{enumerate}
