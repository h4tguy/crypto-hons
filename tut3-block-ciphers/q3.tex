We describe a well-known distinguishing attack that uses 
$\mathcal{O}(2^{n/2})$ blocks of bitstream. (\cite{dist}, p. 130)

Let the ``blocks'' of the bitstream (the contiguous $n$-bit substrings
that would be the output of successive iterations of the block cipher if
the bitstream were an OFB mode keystream)
be $z_1,\dots,z_i,\dots$. Note that
if there are two indices $i<j$, such that $z_i=z_j$, then
also 
\begin{align*}
	z_{i+1}=E_K(z_i)&=E_K(z_j)=z_{j+1} \\
	z_{i+2}=E_K(z_{i+1})&=E_K(z_{j+1})=z_{j+2},
\end{align*} and generally
$z_{i+k}=z_{j+k}$ for any nonnegative integer $k$ (in fact any integer
$k\ge-i$, using decryption to do induction in the negative direction).
Thus, if we find two such indices $i,j$, and a nonnegative integer $k$,
such that $z_i=z_j$, but $z_{i+k} \neq z_{j+k}$, then the sequence
is certainly random, and we predict as such. On the other hand, if we
find $i<j$ with $z_i=z_j$ and $z_{i+k}=z_{j+k}$ for even a few values of
$k$, then the bitstream is almost certainly the output of a block cipher,
since the probability that $z_{i+k}=z_{j+k}$ for two random 
(independently-chosen) blocks
$z_{i+k}$ and $z_{j+k}$ is $1/2^n$. We thus classify the bitstream as
the output of a block cipher in this case.

The only remaining case is that no such $i,j$ are found. We choose
a large enough sample of the bitstream so that this would be unlikely
if the bitstream were random ($\mathcal{O}(2^{n/2})$ blocks, by the
	birthday paradox). Therefore if no such $i,j$ exist, we can be
	reasonably certain the bitstream is the output of a block cipher, and
	classify it as such.
	We can increase the accuracy of our algorithm by sampling a larger
	substring of the bitstream, increasing the probability that a random
	bitstream would contain two equal blocks.
